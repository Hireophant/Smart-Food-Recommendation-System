\section{Giới thiệu} 

\subsection{Nội dung}
Trong bối cảnh du lịch ngày càng phát triển, trải nghiệm của du khách vẫn gặp nhiều hạn chế 
như thiếu tính cá nhân hoá, khó tìm được món ăn phù hợp khẩu vị, và không được hỗ trợ bởi 
các công cụ thông minh trong quá trình khám phá. 
"Du lịch thông minh" là xu hướng quan trọng nhằm cải thiện hành trình, hỗ trợ ra quyết 
định và nâng cao sự hài lòng của người dùng thông qua công nghệ và trí tuệ nhân tạo.

Dựa trên định hướng đó, nhóm thực hiện đề tài:  
\textbf{Hệ thống gợi ý quán ăn cá nhân hoá cho du khách dựa trên vị trí và khẩu vị}.  
Hệ thống tập trung giải quyết vấn đề thường gặp của du khách khi đến một khu vực mới—không 
biết quán ăn nào phù hợp khẩu vị, món muốn thử hoặc chế độ ăn (chay/mặn, cay/ít cay, ít dầu mỡ).  
Các nền tảng hiện nay như Google Maps hay Foody chỉ trả về danh sách nhà hàng gần vị trí 
nhưng chưa hỗ trợ lọc theo khẩu vị chi tiết, cũng như không hiểu ngôn ngữ tự nhiên dạng:
\begin{quote}
    ``Tôi muốn ăn đồ chay cay nhẹ gần hồ Hoàn Kiếm'' \\
    ``Tìm quán hải sản không quá mặn ở Nha Trang''
\end{quote}

Hệ thống nhóm xây dựng hoạt động dưới dạng ứng dụng desktop, cho phép:
\begin{itemize}
    \item đánh giá vị trí người dùng qua toạ độ (lat, lon);
    \item gợi ý quán ăn dựa trên khẩu vị, món yêu cầu và ngân sách;
    \item hỗ trợ nhập liệu bằng biểu mẫu hoặc chatbot xử lý ngôn ngữ tự nhiên;
    \item khai thác dữ liệu nhà hàng từ OpenStreetMap qua Overpass API và Nominatim.
\end{itemize}

Toàn bộ mô hình được thiết kế theo quy trình tư duy tính toán: phân tích bài toán, tách nhỏ 
(thành phần dữ liệu và các bộ lọc), trừu tượng hoá các thuộc tính quan trọng, thiết kế thuật toán 
lọc–xếp hạng, biểu diễn bằng lưu đồ và mô phỏng thử nghiệm theo yêu cầu của môn học.

\subsection{Mục tiêu}
Mục tiêu của đề tài bao gồm:
\begin{enumerate}
    \item Xây dựng một mô hình gợi ý quán ăn cá nhân hoá dựa trên vị trí và khẩu vị người dùng.
    \item Ứng dụng tư duy tính toán (Computational Thinking) trong toàn bộ quy trình: 
        phân tích, phân rã, trừu tượng hoá, thiết kế thuật toán và mô phỏng.
    \item Kết hợp các công nghệ phù hợp như API bản đồ (Overpass, Nominatim) 
        và NLP cho chatbot để xử lý đầu vào dạng ngôn ngữ tự nhiên.
    \item Tạo ra giao diện ứng dụng desktop giúp người dùng nhập thông tin dễ dàng 
        và nhận gợi ý trực quan.
    \item Đánh giá hệ thống theo tiêu chí: độ phù hợp gợi ý, tốc độ phản hồi, 
        khả năng hiểu truy vấn và mức độ giải quyết pain points thực tế.
    \item Rèn luyện khả năng làm việc nhóm, thiết kế hệ thống và trình bày báo cáo kỹ thuật.
\end{enumerate}

\subsection{Lời cảm ơn}
Nhóm xin gửi lời cảm ơn chân thành đến giảng viên hướng dẫn \textit{ThS. Trần Duy Quang}, 
người đã cung cấp định hướng, tài liệu và phản hồi giúp nhóm hoàn thiện đề tài đúng mục tiêu 
của học phần. Nhóm cũng cảm ơn các thầy cô bộ môn đã xây dựng nội dung học phần 
``Tư Duy Tính Toán'' và tài liệu hướng dẫn dự án ``Smart Tourism System'', tạo nền tảng để 
nhóm vận dụng công nghệ vào giải quyết bài toán thực tế.  
Cuối cùng, nhóm ghi nhận sự hỗ trợ của các công cụ học thuật và tham khảo như ChatGPT trong 
việc chỉnh sửa diễn đạt và tối ưu bố cục báo cáo, nhưng toàn bộ nội dung phân tích và triển khai 
đều do nhóm trực tiếp thực hiện và chịu trách nhiệm.
