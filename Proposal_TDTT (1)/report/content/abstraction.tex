\section{Trừu tượng hoá (Abstraction)}

Để giảm độ phức tạp, hệ thống chỉ giữ lại những thuộc tính cốt lõi nhất ảnh hưởng đến 
việc gợi ý nhà hàng. Các chi tiết không cần thiết được loại bỏ nhằm tối ưu hoá tính toán 
và mô hình dữ liệu.

\subsection{Thực thể Nhà hàng (Restaurant)}
Mỗi nhà hàng được mô hình hoá bởi các thuộc tính:
\begin{itemize}
    \item toạ độ (lat, lon),
    \item tên,
    \item loại ẩm thực (cuisine),
    \item khoảng giá,
    \item tập nhãn khẩu vị (tags),
    \item các thuộc tính phụ (rating, địa chỉ).
\end{itemize}

\subsection{Thực thể Người dùng (User)}
Các thuộc tính cần thiết:
\begin{itemize}
    \item Vị trí hiện tại hoặc vị trí muốn tìm quán.
    \item Sở thích món ăn (món Việt, món Hàn, phở, bún bò…), khẩu vị.
    \item ngân sách mong muốn,
    \item bán kính tìm kiếm.
\end{itemize}
\subsection{Recommendation System}
\begin{itemize}
\item \textbf{Nhận và phân tích đầu vào}: Chuyển User Profile thành một tập đặc trưng chuẩn hóa, bao gồm loại món, bán kình tìm kiếm, khẩu vị, ...
\item \textbf{Matching – ánh xạ giữa đặc trưng người dùng và đặc trưng nhà hàng}: So sánh vector người dùng với vector các nhà hàng thông qua một hàm đánh giá trừu tượng (tiêu chí: phù hợp khẩu vị, đánh giá độ phổ biến, khoảng cách.)
\item \textbf{Ranking – xếp hạng theo độ phù hợp}: Tạo một danh sách gồm các nhà hàng được sắp xếp theo điểm phù hợp giảm dần.
\item \textbf{Hàm tính toán "điểm ưu tiên" của hệ thống}: $f(User,Restaurants) \Rightarrow SortedList(Restaurants)$ \\
Trong đó: 
\begin{itemize}
    \item $User$: vector thuộc tính mô tả người dùng.
    \item $Restaurants$: tập các nhà hàng dưới dạng vector thuộc tính.
    \itemn $f$: hàm gợi ý (Recommendation Function) — có thể là thuật toán, ML model, rule-based, hoặc hybrid.
    \item $SortedList(Restaurants)$: danh sách nhà hàng đã được xếp hạng theo mức độ phù hợp.
\end{itemize}
\end{itemize}
