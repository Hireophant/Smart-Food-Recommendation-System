\section{Phân rã vấn đề (Decomposition)}

Bài toán được chia thành nhiều mô-đun nhỏ nhằm đơn giản hoá quy trình xử lý, tách biệt trách 
nhiệm và giúp hệ thống dễ phát triển, kiểm thử và mở rộng. Dưới đây là các thành phần chính 
và nhiệm vụ tương ứng.

\subsection{Thu thập và xử lý thông tin người dùng}
\textbf{Mục tiêu:} Chuẩn hoá toàn bộ dữ liệu đầu vào của người dùng trước khi đưa vào mô-đun gợi ý (rõ ràng, đầy đủ và dễ xử lý cho AI).

\textbf{Các nhiệm vụ con:}
\begin{itemize}
    \item Thu nhận vị trí người dùng: nhập tay hoặc chọn từ bản đồ (lat, lon).
    \item Thu nhận khẩu vị và yêu cầu món ăn: chay/mặn, độ cay, mức ngọt, loại cuisine.
    \item Xử lý dữ liệu nhập qua chatbot: tách từ khoá về món ăn, mức cay, địa điểm.
    \item Tiếp nhận các bộ lọc tuỳ chọn: bán kính tìm kiếm, giá, đánh giá.
    \item Chuẩn hoá đầu vào thành cấu trúc JSON cố định
\end{itemize}

\subsection{Xử lý và quản lý dữ liệu nhà hàng}
\textbf{Mục tiêu:} Tạo nguồn dữ liệu quán ăn sạch, thống nhất, dễ truy vấn.

\textbf{Các nhiệm vụ con:}
\begin{itemize}
    \item Kết nối API bản đồ (Overpass API) để lấy danh sách nhà hàng quanh vị trí người dùng.
    \item Làm sạch dữ liệu: loại trùng lặp, chuẩn hoá tên, định dạng lại cuisine/tags.
    \item Gắn nhãn khẩu vị (vegetarian, spicy, sweet, healthy) nếu thông tin có sẵn.
    \item Gắn tag cho nhà hàng: "chay", "đồ cay", "bình dân", "gia đình", "đồ Hàn", ...
\end{itemize}

\subsection{Phân tích yêu cầu và hiểu ngữ nghĩa (NLP Processing)}
\textbf{Mục tiêu:} Biến ngôn ngữ tự nhiên → dữ liệu chuẩn để Recommendation Engine xử lý.

\textbf{Các nhiệm vụ con:}
\begin{itemize}
    \item Nhận dạng ý định (intent): tìm nhà hàng, xem món ăn, hỏi gợi ý.
    \item Trích xuất thực thể (entity extraction): món ăn, đặc điểm khẩu vị, địa điểm.
    \item Chuẩn hoá từ khoá: xử lý đồng nghĩa, lỗi chính tả.
    \item Trả về tập thuộc tính chuẩn hoá để đưa vào mô-đun gợi ý.
\end{itemize}

\subsection{Quản lý hồ sơ người dùng (User Profile Management)}
\textbf{Mục tiêu:} Lưu và học từ các hành vi của người dùng để cải thiện độ chính xác của gợi ý.

\textbf{Các nhiệm vụ con:}
\begin{itemize}
    \item Tạo hồ sơ người dùng (diet, sở thích vị, món yêu thích).
    \item Lưu lịch sử tìm kiếm hoặc gợi ý đã chọn.
    \item (Tuỳ chọn) Điều chỉnh trọng số sở thích dựa trên hành vi người dùng.
    \item Hỗ trợ chỉnh sửa/xoá dữ liệu cá nhân (bảo mật, quyền riêng tư).
\end{itemize}

\subsection{Bộ máy gợi ý (Recommendation Engine)}
\textbf{Mục tiêu:} Tính toán và xếp hạng danh sách quán ăn phù hợp nhất.

\textbf{Các nhiệm vụ con:}
\begin{itemize}
    \item Lọc theo bán kính, mức giá, đánh giá (manual filtering).
    \item So khớp khẩu vị người dùng với tag của nhà hàng.
    \item Tính điểm phù hợp (matching score) dựa trên: khẩu vị, khoảng cách, giá.
    \item Xếp hạng và trả về danh sách top-$k$ kết quả.
    \item Học từ lịch sử người dùng → gợi ý các quán tương tự đã yêu thích
\end{itemize}

\subsection{Hiển thị giao diện và kết quả (UI/UX)}
\textbf{Mục tiêu:} Giúp người dùng xem và tương tác với kết quả gợi ý.

\textbf{Các nhiệm vụ con:}
\begin{itemize}
    \item Hiển thị danh sách quán ăn (tên, địa chỉ, giá, khoảng cách).
    \item Tích hợp bản đồ với các marker vị trí quán.
    \item Tương tác chatbot: hỏi món, hỏi khẩu vị, thay đổi yêu cầu.
    \item Cho phép lọc lại kết quả theo chế độ thủ công.
\end{itemize}

\subsection{7. Đánh giá và phản hồi (Evaluation \& Feedback)}
\textbf{Mục tiêu:} Cải thiện hệ thống theo phản hồi của người dùng.

\textbf{Các nhiệm vụ con:}
\begin{itemize}
    \item Ghi nhận hành vi người dùng: nhấp xem chi tiết, chọn quán.
    \item Thu thập phản hồi: ``gợi ý này có phù hợp không?''.
    \item Điều chỉnh thuật toán hoặc gợi ý trong lần sử dụng tiếp theo.
\end{itemize}

\subsection{8. Xử lý ngoại lệ và rủi ro (Exception Handling)}
\textbf{Mục tiêu:} Đảm bảo hệ thống hoạt động đáng tin cậy, an toàn và ổn định.

\textbf{Các nhiệm vụ con:}
\begin{itemize}
    \item Nếu API không phản hồi → sử dụng dữ liệu cache.
    \item Nếu không có kết quả phù hợp → hiển thị quán phổ biến gần đó.
    \item Xác thực đầu vào người dùng để tránh lỗi nhập liệu.
    \item Bảo vệ dữ liệu cá nhân và quyền riêng tư.
\end{itemize}
