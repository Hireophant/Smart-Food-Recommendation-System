\section{Phân tích vấn đề (Problem Analysis)}

\subsection{Định nghĩa bài toán (Problem Definition)}
Mục tiêu là xây dựng \textbf{hệ thống gợi ý quán ăn cá nhân hoá theo vị trí và khẩu vị người dùng}.  
Hệ thống nhận đầu vào:
\begin{itemize}
    \item Toạ độ người dùng hoặc địa chỉ (được geocoding thành lat, lon).
    \item Khẩu vị: chay/mặn, cay/ít cay, ngọt/thanh, healthy, v.v.
    \item (Tùy chọn) Món muốn ăn.
    \item (Tuỳ chọn) Ngân sách, bán kính tìm kiếm, rating.
\end{itemize}

Đầu ra bao gồm danh sách quán ăn phù hợp nhất: tên, địa chỉ, khoảng cách, mức giá, điểm phù hợp 
và toạ độ để điều hướng.

\subsection{Các bên liên quan (Stakeholders)}
\subsubsection*{Primary Stakeholders}
\begin{itemize}
    \item \textbf{Người dùng cuối (du khách / người dân địa phương)}: cần hệ thống giúp lọc nhanh quán ăn phù hợp 
    với khẩu vị cá nhân.
\end{itemize}

\subsubsection*{Secondary Stakeholders}
\begin{itemize}
    \item \textbf{Nhà hàng}: hưởng lợi khi tiếp cận đúng nhóm khách có nhu cầu phù hợp.
\end{itemize}

\subsubsection*{External Stakeholders}
\begin{itemize}
    \item \textbf{Nhà cung cấp dữ liệu (OSM)}: cung cấp dữ liệu vị trí và POI.
\end{itemize}

\subsubsection*{Internal Stakeholders}
\begin{itemize}
    \item \textbf{Nhóm phát triển}: thiết kế hệ thống, xây dựng mô hình lọc, tối ưu UX.
\end{itemize}

\subsection{Phạm vi (Scope)}
\subsubsection*{In-scope}
\begin{itemize}
    \item Gợi ý quán ăn theo vị trí người dùng (lat, lon).
    \item Cá nhân hoá theo khẩu vị: chay/mặn, cay/ít cay, ngọt/thanh, ít dầu mỡ.
    \item Hỗ trợ nhập liệu: form hoặc chatbot.
    \item Lấy dữ liệu nhà hàng qua API hoặc tìm data thủ công.
    \item Triển khai dưới dạng ứng dụng desktop.
\end{itemize}

\subsubsection*{Out-of-scope}
\begin{itemize}
    \item Không hỗ trợ đặt bàn, thanh toán hoặc hệ thống review phức tạp.
    \item Không xây dựng mạng xã hội hoặc chức năng chia sẻ.
    \item Không đảm bảo dữ liệu đầy đủ 100\% trên toàn quốc.
    \item Không sử dụng mô hình AI lớn nội bộ (ưu tiên API hoặc rule-based).
\end{itemize}

\subsection{Các điểm đau của người dùng (Pain Points)}
Dựa trên quan sát hành vi người dùng và phân tích đánh giá trên Google Maps/Foody, nhóm 
xác định các pain points chính sau:
\begin{itemize}
    \item \textbf{Quá nhiều lựa chọn nhưng thiếu cá nhân hoá}: người dùng nhận danh sách dài nhưng không 
    biết quán nào hợp khẩu vị.
    \item \textbf{Khẩu vị cá nhân khó lọc}: các app không có filter cho ``cay/ít cay'', ``chay'', ``healthy'', v.v.
    \item \textbf{Không hiểu tìm kiếm mô tả tự nhiên}: các truy vấn dạng câu nói không được xử lý.
    \item \textbf{Du khách không biết món đặc trưng địa phương}: thiếu thông tin để lựa chọn đúng món/quán.
\end{itemize}

\subsection{Kiểm chứng điểm đau (Pain Point Validation)}

Để đảm bảo các pain point không chỉ xuất phát từ suy đoán chủ quan của nhóm, việc xác thực 
được thực hiện dựa trên \textbf{ba loại nguồn độc lập}, theo đúng nguyên tắc tam giác dữ liệu 
(\textit{triangulation}):

\subsubsection*{(1) Nghiên cứu đối thủ (Competitive Research)}
Nhóm tiến hành khảo sát thử trên các nền tảng phổ biến như Google Maps và Foody. 
Kết quả quan sát:
\begin{itemize}
    \item Google Maps chỉ hỗ trợ các bộ lọc cơ bản (khoảng cách, đánh giá, giá tiền), 
    \textbf{không có bộ lọc khẩu vị} như chay/mặn, độ cay, mức dầu mỡ.
    \item Foody có thể tìm theo ``món ăn'', nhưng \textbf{không thể lọc theo đặc tính khẩu vị cá nhân}, 
    đặc biệt là khẩu vị đặc thù (healthy, ít dầu mỡ, cay nhẹ).
    \item Cả hai nền tảng đều \textbf{không hiểu truy vấn dạng ngôn ngữ tự nhiên}, 
    ví dụ: ``tôi muốn ăn đồ chay cay nhẹ gần bãi biển Mỹ Khê''.
\end{itemize}
Điều này xác nhận pain point về \textbf{thiếu cá nhân hoá khẩu vị} và \textbf{không hiểu intent} là có thật.

\subsubsection*{(2) Quan sát hành vi người dùng (User Behavior Observation)}
Nhóm thực hiện một khảo sát nhỏ dạng quan sát hành vi (\textit{informal observation}) 
trên 8–10 sinh viên trong trường:
\begin{itemize}
    \item Hầu hết người dùng khi tìm quán ăn sẽ mở Google Maps, chọn ``Nearby'' và sau đó
    phải mở từng nhà hàng để xem menu và đánh giá.
    \item Thời gian trung bình để chọn được quán phù hợp là khoảng 10–15 phút.
    \item Nhiều người chia sẻ rằng họ ``không chắc quán đó có phù hợp khẩu vị hay không''
    cho đến khi xem chi tiết từng menu, gây \textbf{quá tải thông tin}.
\end{itemize}
Quan sát này xác nhận pain point về \textbf{tốn thời gian chọn quán} và \textbf{không có hỗ trợ khẩu vị rõ ràng}.

\subsubsection*{(3) Phân tích đánh giá trực tuyến (Online Review Analysis)}
Nhóm tham khảo các bình luận trên Google Maps và Foody, nhận thấy một số mẫu lặp lại:
\begin{itemize}
    \item Người dùng thường phàn nàn về việc ``không đúng khẩu vị'', ví dụ: món quá mặn, quá cay, 
    nhiều dầu, không hợp chế độ ăn.
    \item Người đi du lịch thường bình luận rằng ``không biết món nào nên thử'' hoặc 
    ``khó tìm quán hợp khẩu vị địa phương''.
    \item Nhiều người viết rằng họ phải dựa vào đánh giá ngẫu nhiên và thử vận may, thay vì có công cụ 
    đề xuất chính xác.
\end{itemize}
Những đánh giá này củng cố pain point về \textbf{thiếu thông tin khẩu vị phù hợp cho du khách}.

\subsubsection*{Kết luận Validation}
Việc kết hợp ba nguồn dữ liệu (đối thủ, hành vi người dùng, và đánh giá trực tuyến) cho thấy 
các pain point đã được xác thực bởi nhiều nguồn độc lập và có tính thực tế cao.  
Do đó, vấn đề mà hệ thống hướng đến giải quyết là \textbf{có thật, phổ biến, và chưa được 
giải quyết tốt bởi các giải pháp hiện có}.


\subsection{Ràng buộc (Constraints)}
\begin{itemize}
    \item Ứng dụng chỉ chạy trên desktop.
    \item Phụ thuộc vào dữ liệu OSM và API bản đồ.
    \item Nhiều quán ăn thiếu thông tin chi tiết như menu, mức giá. Nhưng cũng nhiều quán có quá nhiều món trong menu.
    \item Giới hạn tốc độ truy vấn của các API.
\end{itemize}

\subsection{Rủi ro (Risks)}
\begin{itemize}
    \item Dữ liệu nhà hàng không đầy đủ hoặc bị sai.
    \item API timeout nếu truy vấn bán kính lớn.
    \item Chatbot có thể hiểu sai khẩu vị do câu nhập không rõ ràng.
    \item Vấn đề cold start với người dùng mới.
\end{itemize}

\subsection{Tiêu chí thành công (Success Criteria)}
Hệ thống được xem là thành công nếu:
\begin{itemize}
    \item Tối thiểu 70\% người dùng thử nghiệm đánh giá top-3 gợi ý là ``phù hợp'' hoặc ``rất phù hợp''.
    \item Thời gian phản hồi dưới 3 giây cho bán kính tìm kiếm 1km.
    \item Chatbot hiểu đúng ít nhất 80\% truy vấn khẩu vị đơn giản.
    \item Giải quyết ít nhất hai pain point mà ứng dụng bản đồ phổ biến hiện tại chưa giải quyết.
\end{itemize}
